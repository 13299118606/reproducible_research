\published{Exploration Geophysics, 2014, doi:http://dx.doi.org/10.1071/EG14051}

\title{Deblending using a space-varying median filter}
\renewcommand{\thefootnote}{\fnsymbol{footnote}}

\author{Yangkang Chen}

\address{
\footnotemark[1]Bureau of Economic Geology, \\
Jackson School of Geosciences \\
The University of Texas at Austin \\
University Station, Box X \\
Austin, TX 78713-8924 \\
USA \\
}

\lefthead{Chen}
\righthead{Deblending using SVMF}

\maketitle

\begin{abstract}
Deblending is a currently popular method for dealing with simultaneous-source seismic data. Removing blending noise while preserving as much useful signal as possible is the key to the deblending process. In this paper, I propose to use \emph{space-varying median filter} (SVMF) to remove blending noise. I demonstrate that this filtering method preserves more useful seismic reflection than does the conventional version of median filter (MF). In SVMF, I use \emph{signal reliability} (SR) as a reference to pick up the blending spikes and increase the window length in order to attenuate the spikes. When useful signals are identified, the window length is decreased in order to preserve more energy. The SR is defined as the local similarity between the data initially filtered using MF and the original noisy data. In this way, SVMF can be regionally adaptive, instead of rigidly using a constant window length through the whole profile for MF. Synthetic and field-data examples demonstrate excellent performance for my proposed method.
\end{abstract}

%\section{Keywords}
%simultaneous source, deblending, space-varying median filter, signal reliability and local similarity


\section{Introduction}

The technique of simultaneous-source (sometimes called multisource) acquisition involves firing more than one shot at nearly the same time, regardless of source interference. In conventional acquisition, either the temporal shooting intervals (or ignition interval \cite[]{wapenaar20121}) or the spatial sampling intervals are large enough that the crosstalk between successive shots can be left out. The novel simultaneous-source technique can reduce the acquisition period and at the same time can improve data quality by decreasing the spatial-sampling interval \cite[]{berk}. The benefits from simultaneous-source acquisition are compromised by the challenges in removing blending interference.  Because of its economic benefits and technical challenges, this technique has attracted the attention of researchers in both industry and academia \cite[]{moore2008,araz2011,mediandeblend}. 

Two main ways exist to address the challenges of simultaneous-source acquisition. The first way is to use a first-separate and second-process strategy \cite[]{spitz2008,araz2011,abma,yangkang20131}, also known as "deblending" \cite[]{pana1}. The other is to use direct imaging and waveform inversion by applying some constraint to eliminate the artifacts caused by interference \cite[]{yaxun2009,zhiyong2010,daiwei2011,daiwei2012}. Although the direct imaging approach has achieved some encouraging results, the currently preferred way is still to focus on the separation of blended data into individual sources. Currently existing deblending methods fall into two categories. The first is based on filtering \cite[]{gary,mediandeblend}, which treats the deblending problem simply as a noise attenuation problem. This treatment is logical because the blending noise, although coherent in the common-shot domain, has been demonstrated to be incoherent in other domains, such as common-receiver, common-offset, and common-midpoint domains \cite[]{beasley2,berk}. Thus, all the conventional denoising algorithms can be used in the same way. The second category includes methods based on inversion, converting the separation problem into an inversion problem. Because of the ill-posed property of the blending equation, some specific constraints must be added. The inversion problem can be solved either by directly inverting the matrix of the forward modeling operator \cite[]{wapenaar20121,wapenaar20122}, or by using an iterative framework that iteratively estimates the useful signal and subtracts the  blending noise \cite[]{araz2011,araz2012,pana2,yangkang20131}. 
%This type is more like a combination of the filtering and inversion based methods, becasue it uses an iterative way to solve the blending equation and constrain the model by applying a sparsity or coherency promoting filter. 
%Other methods are based on improving


%Different filtering and inversion methods have been used previously to deblend seismic data. Filtering methods utilize the property that the coherency of the simultaneous-source data is not the same in different domains, thus we can get the unblended data by filtering out the randomly distributed blending noise in one special domain, where one source record is coherent and the other is not \cite[]{gary,araz2012,mediandeblend}. Inversion methods treat the separation problem as an estimation problem which aims at estimating the desired unknown unblended data. Because of the ill-posed property of such estimation problems, a regularization term is usually required \cite[]{pana2}. \cite{roald} proposed to distribute all energy in the %simultaneous 
%shot records by reconstructing the individual shot records at their respective locations. \cite{araz2012} introduced an iterative estimation and subtraction scheme that combines the property of filtering and inversion methods and exploits the fact that the characteristics of the blending noise differs in different domains. One choice is to transform seismic data from the common-shot domain to common-receiver, common-offset or common-midpoint domain. \cite{proj} proposed a separation technique called the alternating projection method (APM), which was demonstrated to be robust in the presense of aliasing. 
%However, most of the published methods will either cause heavy signal leakage or cause large iterative computation. New efficient and robust deblending scheme is still demanded.

Median filter (MF) is well-known for its ability to remove spiky noise in a seismic profile after NMO or with relatively flatter events. MF has been successfully utilized in the deblending process \cite[]{mediandeblend,araz2012}. In this paper, I demonstrate the use of a \emph{space-varying median filter} (SVMF) instead of a conventional MF to remove blending noise. I squeeze or stretch the length of conventional MF according to the \emph{signal reliability} (SR) as a signal point. The SR is defined as the local similarity between the data initially filtered using MF and the original noisy data. Different SRs correspond to different window length for filtering. %Another criteria for picking up signal point is by using \emph{signal energy} (SE). The SE is defined as the absolute value of the initially MF filtered data. However, SR is better than SE in detecting signal points in two aspects. 
%The SVMF is a direct extension to the previously proposed 1-D time varying median filter (TVMF) \cite[]{liuyang2009tvmf}, but differs from TVMF in two aspects. On one hand SVMF should always be implemented along the spatial direction and the length of the filtering window varies according to the spatial position. On the other hand, the reference energy can be obtained both empirically and through the computation with respect to the total number of points in processing window. 
I use both synthetic and field blended common-receiver gathers to demonstrate the better energy-preserving property of SVMF. I also use both synthetic and field prestack time-shot-offset (TSO) domain blended data cube to demonstrate the efficient and effective performance of the proposed deblending approach using SVMF.

%This paper is organized as follows: first I introduce the mathematical expressions for both MF and SVMF and give a novel criteria for choosing variable filtering window length. Second, I review another criteria for choosing variable filtering window length used previously in the literature and compare the performances of the two criterion in picking up useful signal and choosing variable filter length in two different cases: clean and noisy synthetic data. Third, I give the implementation steps of the proposed SVMF and clarify the implementation difference of MF-based filtering between exploration geophysics and image processing fields. Fourth, I demonstrate the performance of the proposed SVMF in deblending simultaneous-source data in the common-receiver domain using both synthetic and field data examples. Finally, I demonstrate the performance of the proposed SVMF in deblending pre-stack marine-streamer simultaneous-source data in time-midpoint-offset (TMO) domain. 

\section{Method}
\subsection{Median filter}
Conventional MF is based on a scalar-value sorting process. When a set of scalars is sorted to form an ascending or descending sequence, the middle value is chosen as standard for this sequence. In signal-processing or geophysical data analysis fields, this filter is commonly used to remove spiky noise. The more general mathematical formulation of a MF is given as:
\begin{equation}
\label{eq:mf1}
\hat{u}_{i,j}=\arg\min_{u_m\in U_{i,j}}\sum_{l=1}^{L}\Arrowvert u_m-u_l \Arrowvert_p,
\end{equation}
where $\hat{u}_{i,j}$ is the output value for location $x_{i,j}$; $U_{i,j}=\{u_1,u_2,\cdots,u_L\}$, $i$ and $j$ are the position indices in a 2-D profile; and $l$ and $m$ are both indices in the filtering window. $L$ is the length of the filtering window and $p$ denotes $L_p$ norm. Commonly, $p=1$ corresponds to a standard MF. 

\subsection{Space-varying median filter}
For SVMF, $L$ becomes $L_{i,j}$, varying with respect to location $x_{i,j}$. The new filtering expression is:
\begin{equation}
\label{eq:svmf1}
\hat{v}_{i,j} = \arg\min_{v_{m}\in U_{i,j} }\sum_{l=1}^{L_{i,j}} \Arrowvert v_{m} -v_l \Arrowvert_p,
\end{equation}
where $\hat{v}_{i,j}$ is the output value for location $x_{i,j}$ after applying a SVMF, $U_{i,j}=\{v_1,v_2,\cdots,v_{L_{i,j}}\}$. The filter length $L_{i,j}$ can be chosen through the following empirical criteria:
\begin{equation}
\label{eq:svmf2}
L_{i,j}=\left\{\begin{array}{ll}
L+l_1,\quad 0\quad \le |s^L_{i,j}|\le0.15s_{max} \\
L+l_2,\quad 0.15s_{max} < |s^L_{i,j}|<0.25s_{max} \\
L,\quad \quad\quad 0.25s_{max} \le |s^L_{i,j}|\le0.75s_{max} \\
L-l_3,\quad 0.75s_{max}<|s^L_{i,j}| < 0.85s_{max}\\
L-l_4,\quad 0.85s_{max}\le|s^L_{i,j}| \le s_{max}
\end{array}\right.,
\end{equation}
where $l_1$,$l_2$,$l_3$,$l_4$ are predefined parameters corresponding to the increments or decrements for the length of filter window \new{and are generally chosen as 4,2,2,4 in default, respectively}; $s^L_{i,j}$ is the \emph{signal reliability} (SR), which can be defined as the local similarity \cite[]{fomel2007localattr} between the initially filtered data $u^L_{i,j}$ with a window length $L$ and the original data $u_{i,j}$ for point $x_{i,j}$:

\begin{equation}
\label{eq:simi}
s^L_{i,j} = \mathbf{S} [u^L_{i,j}, u_{i,j}] 
\end{equation}
Here, $\mathbf{S} [\mathbf{x},\mathbf{y}]$ denotes the local similarity between $\mathbf{x}$ and $\mathbf{y}$, and $s_{max}$ denotes the maximum value of the similarity map. Appendix A gives a short review of local similarity.

\subsection{Comparison between signal reliability based SVMF and signal energy based SVMF}
Another criteria for selecting the variable window length $L_{i,j}$ is by \emph{signal energy} (SE), as introduced in the \emph{time-varying median filter} (TVMF) framework \cite[]{liuyang2009tvmf}:
\begin{equation}
\label{eq:svmf3}
L_{i,j}=\left\{\begin{array}{ll}
L+l_1,\quad 0\quad \le |e^L_{i,j}|\le T/2 \\
L+l_2,\quad T/2\quad <|e^L_{i,j}|\le T \\
L-l_3,\quad T \quad \le|e^L_{i,j}| < 2T\\
L-l_4,\quad |e^L_{i,j}| \ge 2T
\end{array}\right.,
\end{equation}
where $e^L_{i,j}$ is the \emph{signal energy} (SE), which can be defined as the absolute value of the initially filtered data $u^L_{i,j}$ with a window length $L$ for point $x_{i,j}$. $T$ is a threshold value, and can be calculated by:
\begin{equation}
\label{eq:threshold}
T=\frac{1}{N_x\times N_t}\sum_{i=1}^{N_x}\sum_{j=1}^{N_t} |e^L_{i,j}|.
\end{equation}
Here, $N_x$ and $N_t$ denote the number of spatial and temporal samples.

The SE based SVMF differs from the SR based SVMF in that the former uses SE as a reference to pick up useful signal while the latter uses SR to pick up useful signal. Figures \ref{fig:simi1} and \ref{fig:energy1} give a comparison between the normalized SR and SE maps for the synthetic example as shown in Figure \ref{fig:data1,datas}. %corresponding to equations \ref{eq:svmf2} and \ref{eq:svmf3}, respectively. 
Figures \ref{fig:L1} and \ref{fig:L1-tsmf} gives the corresponding filter length map. As we can see, the SE map has a higher resolution while SR map gives a smoother result. However, the SE map gives some "fake" points, as indicated by arrows. These "fake" points come from the remnant blending noise. Because the initial MF can not remove all the blending noise, some noise points will have large amplitude thus will be picked up by the SE criteria. These "fake" points may result in very short filter length even for noisy area, as indicated by the arrows in Figure \ref{fig:L1-tsmf}. Because of an embedded smoothing function when calculating the local similarity, those remnant noise points will be smoothed and thus show small amplitude in the SR map. The SR map can get much better result when the seismic profile contains some ambient random noise. Because the initial MF in the SVMF will harm the random ambient noise rather than removing them, the amplitude properties of random ambient noise can not be changed too much, but the similarity properties will be changed a lot. Figure \ref{fig:simi1-n,energy1-n,L1-n,L1-tsmf-n} gives the comparison between SR and SE maps and their corresponding filter length maps after adding some ambient Gaussian white noise to the seismic data. In this case, the SR can still capture the useful events well while the SE can not. The corresponding variable filter length map thus gives a more plausible reference. 

\inputdir{test}
\multiplot{4}{simi1,energy1,L1,L1-tsmf}{width=0.46\textwidth}{(a) SR map and (b) SE map corresponding to the data shown in Figure \ref{fig:data1,datas}. (c) Filter length map using SR. (d) Filter length map using SE.}

\inputdir{testnoise}
\multiplot{4}{simi1-n,energy1-n,L1-n,L1-tsmf-n}{width=0.46\textwidth}{(a) SR map and (b) SE map for noisy data. (c) Filter length map using SR. (d) Filter length map using SE.}

 %In order to be compared with the SR based criterion fairly, I modified the SE based criterion as the following formulation in order to match the formulation as shown in equation \ref{eq:svmf2}:
%\begin{equation}
%\label{eq:svmf2}
%L_{i,j}=\left\{\begin{array}{ll}
%L+l_1,\quad 0\quad \le |e^L_{i,j}|\le0.15e_{max} \\
%L+l_2,\quad 0.15e_{max} < |e^L_{i,j}|<0.25e_{max} \\
%L,\quad \quad\quad 0.25s_{max} \le |e^L_{i,j}|\le0.75e_{max} \\
%L-l_3,\quad 0.75e_{max}<|e^L_{i,j}| < 0.85e_{max}\\
%L-l_4,\quad 0.85e_{max}\le|e^L_{i,j}| \le e_{max}
%\end{array}\right.,
%\end{equation}
%where $e^L_{i,j}$ is the \emph{signal energy} (SE), which can be defined as the absolute value of the initially filtered data $u^L_{i,j}$ with a window length $L$ for point $x_{i,j}$. $e_{max}$ denotes the maximum value of SE.


\subsection{Implementation steps of SVMF}
Unlike 2D signal-processing field, where the signal is multi-dimensionally coherent, geophysical data is energy-focusing only spatially . Due to the temporal sparseness of the property, the useful signal takes a spike-like form. This spatial coherence makes it necessary to take a conventional MF along the spatial direction. Besides, the local slope of an event should be small in order to ensure a small energy loss. In the case of dipping events, a multidimensional MF can be used as a substitute of the conventional MF \cite[]{mediandeblend}, or in the other way, one can let the length of filtering window be smaller. However, reducing the length of the filtering window reduces the ability of an MF to remove spiky noise commensurately. Balancing removal of spiky noise while minimizing energy loss is always a compromise. The SVMF utilizes a two-step strategy: first, use MF to coarsely filter the data and obtain a calculation of \emph{signal reliability} (SR), and then use an adaptive MF filter with window length varying according to the difference with respect to SR. \new{The initial constant filter length is chosen so that most of the blending noise can be removed regardless of a small loss of useful signals. Empirically, the initial filter length can be between 7 and 11.} In both steps, the MF is implemented along the spatial direction. Figure \ref{fig:huos,huos-tmf,huos-xmf,huos-svmf} shows a filtering comparison using different kinds of filters. The data is similar to that used in \cite{mediandeblend}. To be effective, MF should only be implemented along the spatial direction, and by using SVMF, the dipping events can be preserved to a large extent. The algorithms steps of SVMF can be summarized as follows:
\begin{enumerate}
\item Apply the first MF using constant filter length.
\item Compute the SR map by computing the local similarity between the data initially filtered using MF and the original seismic data.
\item Compute the map of variable filter length.
\item Apply the secondary MF using the variable filter length.
\end{enumerate}


\inputdir{timespatial}
\multiplot{4}{huos,huos-tmf,huos-xmf,huos-svmf}{width=0.46\textwidth}{Comparison of different kinds of MF. (a) Original noisy data. (b) Implementing MF along temporal direction. (c) Implementing MF along spatial direction. (d) Implementing SVMF with two-step MF along spatial direction.}

%\inputdir{timespatial}


%\multiplot{4}{huos,huos-tmf,huos-xmf,huos-svmf}{width=0.23\columnwidth,height=0.470\columnwidth}{Demonstration for different kinds of MFs. (a) Original noisy data. (b) Implementing MF along temporal direction. (c) Implementing MF along spatial direction. (d) Implementing SVMF by two steps.}


\section{Examples}
\subsection{Deblending in common-receiver domain}
The first example is a synthetic common-receiver gather (CRG). \new{The peak frequency is 30 Hz. The temporal and spatial samplings are 4 ms and 10 m, respectively. There are 512 time samples in this data and there are 256 traces in this data that correspond to 256 different shots.} I blended it with another CRG using the random dithering method described by \cite{yangkang20131}. Being concise, I don't display the other CRG. The unblended and blended data are shown in Figures \ref{fig:data1} and \ref{fig:datas}, respectively. I used MF and SVMF to denoise the blended data (Figure \ref{fig:datas}); I then computed their corresponding noise sections (difference between the blended and deblended sections). The deblending results are shown in Figure \ref{fig:deblended1mf,deblended1svmf,diff1mf,diff1svmf}. While the general deblending effects were similar, it is clear that MF caused much greater loss of energy in the noise sections (see Figures \ref{fig:diff1mf} and \ref{fig:diff1svmf}) than SVMF. \new{The initial filter length for this example is set to be 7. The increments/decrements $l_i$ are chosen as the default values.} Figure \ref{fig:simi1} shows the map of SR. Figure \ref{fig:L1} shows the map of variable filter length for the whole profile. From the scalebar of the map, we can see that the filter length is chosen by the proposed criterion from 3 points to 11 points. By comparing the SR with the different threshold, I was able to identify the signal event and then squeeze the filter length to a safe level: 3 points. For those points which I defined as noise points, the filter length was stretched to the most dangerous level: 11 points. 

The second example is a marine-field CRG. \new{The temporal sampling is 4 ms. There are 4001 time samples and 201 traces in this CRG.} I used the same blending approach as in the first example. The unblended and blended data are shown in Figures \ref{fig:fairunblended2} and \ref{fig:fairblended2}. While both MF and SVMF were able to effectively remove most of the strong spike-like blending noise, SVMF resulted in a much better preservation of the useful signal. The deblending results of MF and SVMF and their corresponding noise sections are shown in Figure \old{\ref{fig:fairunblended2,fairblended2}}\new{\ref{fig:fairdeblended2mf,fairdeblended2svmf,diff2mfframe,diff2svmfframe}}. For better comparison, two zoomed sections corresponding to the two frame boxes shown in Figure \old{\ref{fig:fairunblended2,fairblended2}} \new{\ref{fig:fairdeblended2mf,fairdeblended2svmf,diff2mfframe,diff2svmfframe}} are shown in Figure \ref{fig:diff2mfzoom,diff2svmfzoom}. SVMF clearly causes less loss of useful signal. \new{The initial filter length for this example is set to be 9. The increments/decrements $l_i$ are chosen as the default values.} The maps of SR and variable filter length are shown in Figures \ref{fig:fairsimi2} and \ref{fig:fairL2}. As indicated by the scalebar of the maps, filter length ranged from 5 points to 11 points. 

\inputdir{test}
\multiplot{2}{data1,datas}{width=0.46\textwidth}{Synthetic data examples. (a) Unblended common-receiver-domain data. (b) Blended data.}
\multiplot{4}{deblended1mf,deblended1svmf,diff1mf,diff1svmf}{width=0.46\textwidth}{Demonstration of deblending effects for blended synthetic data. (a)Deblended using MF. (b)Deblended using SVMF. (c)Noise section using MF. (d)Noise section using SVMF.}
%\multiplot{2}{simi1,L1}{width=0.46\textwidth}{(a) Map of SR for synthetic data. (b) Map of variable filter length for synthetic data. }

\inputdir{fairfield}
\multiplot{2}{fairunblended2,fairblended2}{width=0.46\textwidth}{Marine field data examples. (a) Unblended common-receiver-domain data. (b) Blended data.}

\multiplot{4}{fairdeblended2mf,fairdeblended2svmf,diff2mfframe,diff2svmfframe}{width=0.46\textwidth}{Demonstration of deblending effects for blended marine field data. (a) Deblended using MF. (b) Deblended using SVMF. (c) Noise section using MF. (d) Noise section using SVMF. \new{The green frame boxes are zoomed in Figure \ref{fig:diff2mfzoom,diff2svmfzoom} for better comparison.}}

\multiplot{2}{diff2mfzoom,diff2svmfzoom}{width=0.46\textwidth}{Zoomed noise sections \new{(corresponding to the two green frame boxes as shown in Figure \ref{fig:fairdeblended2mf,fairdeblended2svmf,diff2mfframe,diff2svmfframe})}. (a) Zoomed noise section corresponding to MF. (b) Zoomed noise section corresponding to SVMF.}

\multiplot{2}{fairsimi2,fairL2}{width=0.46\textwidth}{(a) Map of SR for field data. (b) Map of variable filter length for field data.}

\subsection{Deblending in time-midpoint-offset (TMO) domain}

\inputdir{class}
\multiplot{6}{data-csg,data-b,datas-csg,data-b-svmf-cmg,data-b-svmf-csg,data-b-svmf-cog}{width=0.42\textwidth}{(a) Synthetic data in TSO domain. (b) Blended data in TMO domain. (c) Blended data in TSO domain. (d) Deblended data by applying SVMF in CMG. (e) Deblended data in TSO domain corresponding to (d). (f) Deblended data by applying SVMF in COG. }

\multiplot{2}{data-b-svmf-cmg-zoom,data-b-svmf-cog-zoom}{width=0.45\textwidth}{Zoomed COG (time: 2 s -4 s, midpoint: 3.0 km - 5.5 km, half-offset: 0.5 km). (a) Applying SVMF in CMG (from Figure \ref{fig:data-b-svmf-cmg}). (b) Applying SVMF in COG (from Figure \ref{fig:data-b-svmf-cog}). }


The third example is a synthetic prestack time-shot-offset (TSO) domain data cube (Figure \ref{fig:data-csg,data-b,datas-csg,data-b-svmf-cmg,data-b-svmf-csg,data-b-svmf-cog}). \new{There are 1501 time samples in this synthetic example and the temporal sampling is 4 ms. The peak frequency is 10 Hz. There are 251 shots and 51 receivers in this example. The shot and receiver intervals are both 50 m.} I blended the data by simulating a \new{independent} marine-streamer \old{independent-}\new{simultaneous} shooting (IMSSS) acquisition with two sources. \new{The demonstration of the IMSSS is shown in Figure \ref{fig:streamer-demo}. Note that the number of simultaneous sources need not be limited to four. The shooting vessels shoot independently as in the conventional way and obtain their own data. With $N$ sources, the efficiency can be increased by $N$ times. The increased efficiency, however, is compromised by interference among different sources. } The unblended and blended data are shown in Figures \ref{fig:data-csg} and \ref{fig:datas-csg}, respectively. Figure \ref{fig:datas-csg} shows that the blending interference turns to be coherent in common-shot gather (CSG). After transforming the data from CSG to common-midpoint gather (CMG), the interference turns to be incoherent spike-like noise. I applied SVMF to both CMG and common-offset gather (COG) in the TMO domain; the results are shown in Figures \ref{fig:data-b-svmf-cmg} and \ref{fig:data-b-svmf-csg}, respectively. In order to maximize the effectiveness of SVMF, I applied a normal move-out (NMO) with an approximate velocity in CMG, in order to make the events flatter. However, this strategy was not feasible in COG. Thus, filtering in CMG will preserve more useful signal than filtering in COG. Figure \ref{fig:data-b-svmf-cmg-zoom,data-b-svmf-cog-zoom} shows a comparison between two zoomed COG from Figures \ref{fig:data-b-svmf-cmg} and \old{\ref{fig:data-b-svmf-cmg}}\new{\ref{fig:data-b-svmf-cog}}, respectively. The comparison confirm the fact that applying SVMF in CMG will preserve more more useful signal that applying SVMF in COG. Figure \ref{fig:data-b-svmf-csg} shows the deblended data in the TSO domain by filtering in CMG. Comparing the deblended data show in Figure \ref{fig:data-b-svmf-csg} with blended data shown in Figure \ref{fig:datas-csg}, it is clear that the deblending has been successful. \new{The initial filter length involved in SVMF for this example is chosen as 7. The increments/decrements $l_i$ are chosen as the default values.}

\inputdir{bei}
\multiplot{6}{gulf-csg,gulf-b,gulfs-csg,gulf-b-svmf-cmg,gulf-b-svmf-csg,gulf-b-svmf-cog}{width=0.42\textwidth}{(a) Marine field data in TSO domain. (b) Blended data in TMO domain. (c) Blended data in TSO domain. (d) Deblended data by applying SVMF in CMG. (e) Deblended data in TSO domain corresponding to (d). (f) Deblended data by applying SVMF in COG. }

\multiplot{2}{gulf-b-svmf-cmg-zoom,gulf-b-svmf-cog-zoom}{width=0.45\textwidth}{Zoomed COG (time: 2 ms - 3 ms, midpoint: 11.725 km - 13.735 km, half-offset: 0.635 km). (a) Applying SVMF in CMG (from Figure \ref{fig:gulf-b-svmf-cmg}). (b) Applying SVMF in COG (from Figure \ref{fig:gulf-b-svmf-cog}). }


The fourth example is a field prestack time-shot-offset (TSO) domain data cube (Figure \ref{fig:gulf-csg,gulf-b,gulfs-csg,gulf-b-svmf-cmg,gulf-b-svmf-csg,gulf-b-svmf-cog}). \new{There are 1000 time samples in this synthetic example and the temporal sampling is 4 ms. There are 291 shots and 48 receivers in this example. The shot and receiver intervals are both 33.5 m.} The blended acquisition is the same as the third example. In this case, the blending noise in CMG turns to be string-like noise. However, the string-like noise appears as spike-like noise along spatial direction. Thus, we can still use the proposed SVMF to remove the blending noise. The unblended data, blended data, deblended data in different domains are all shown in Figure \ref{fig:gulf-csg,gulf-b,gulfs-csg,gulf-b-svmf-cmg,gulf-b-svmf-csg,gulf-b-svmf-cog}. The layout of different figures is exactly the same as the third example. Figure \ref{fig:gulf-b-svmf-cmg-zoom,gulf-b-svmf-cog-zoom} shows the comparison between two zoomed COG corresponding to different filtering gathers. Figure \ref{fig:gulf-b-svmf-cmg-zoom} preserves more small features compared with Figure \ref{fig:gulf-b-svmf-cmg-zoom}, confirming the fact that the proposed approximate NMO can help preserve more useful energy. Comparing the deblended data show in Figure \ref{fig:gulf-b-svmf-csg} with blended data shown in Figure \ref{fig:gulfs-csg}, it is clear that the deblending for the field data example is also successful. \new{The initial filter length involved in SVMF for this example is chosen as 5 to be more conservative. The increments/decrements $l_i$ are chosen as the default values.}

  %The non-iterative deblending process is much more efficient than the iterative deblending approaches \cite[]{araz2011,yangkang20131}. For better comparison, the left column in Figure \ref{fig:gulf-csg,gulf-b,gulfs-csg,gulf-b-svmf-cmg,gulf-b-svmf-csg,gulf-b-svmf-cog} are in shot domain, and the right column in Figure \ref{fig:gulf-csg,gulf-b,gulfs-csg,gulf-b-svmf-cmg,gulf-b-svmf-csg,gulf-b-svmf-cog} are all in midpoint domain.  %When data volumes becomes very large, the iterative deblending approaches can not be afforded. 

\inputdir{.}
\plot{streamer-demo}{width=0.8\textwidth}{Demonstration of independent marine-streamer simultaneous shooting (IMSSS) acquisition with four sources.}

%\section{Discussions and Conclusions}
%The ability of MF to attenuate high-amplitude spiky noise is nearly irreplaceable. Although strictly the blending noise is rather wavelet-like than spike-like noise, the high-amplitude peaks appears similarly to spiky noise along the spatial dimension. 
%Thus they can be removed by the proposed SVMF while leaving the small-amplitude parts to be filtered by some other effective filters like $f-x$ deconvolution or $f-k$ filter \cite[]{araz2012}. For simpler seismic profiles, utilizing the SVMF alone is adequately to fulfill the requirements (e.g., Figure \ref{fig:fairdeblended2svmf}). Even in complex profiles, the proposed SVMF can also find its position. It can be used as a preprocessing tool before using the iterative framework to better estimate the useful signal in order to be faster converged and more stable. 

%I have proposed an effective deblending method using SVMF. The SVMF is implemented through two steps, first applying a MF along the spatial direction and computing the SR among the whole processing window, then applying a variable-window-length MF also along the spatial direction. In simpler seismic profiles with relative flatter events, I propose to use SVMF as a whole deblending tool, but in more complex profiles, I propose to use SVMF as a preprocessing tool followed by existing iterative algorithms. %Future research topics include embedding MF or SVMF into iterative deblending procedures to deal with more complex data, finding more reasonable reference criteria for variable-window-length MF (VWLMF) in the framework of SVMF instead of using that proposed by \cite{liuyang2009tvmf} (such as using the local similarity between the initial profile and the first MF-filtered profile for picking up useful signal points) and developing more robust type of MF.

\section{Conclusions}
I have proposed a novel non-iterative approach for deblending simultaneous-source data. In order to utilize the strong ability of median filter (MF) to remove spike-like blending noise and to solve the problem of MF in harming useful signal, I propose a \emph{space-varying median filter} (SVMF). The SVMF can adapt to different regions with varying window length. The proposed SVMF can detect the useful signal and blending noise by \emph{signal reliability} (SR), defined as the local similarity between the data initially filtered using MF and the original noisy blended data, and then squeeze and stretch the filtering window according to SR. The SVMF can be used in common-receiver gather, common-midpoint gather, and common-offset gather, where the blending noise appears as spike-like noise along the spatial dimension. Synthetic and field data examples show that the proposed deblending approach can efficiently and effectively separate the blending data while preserving the useful signal.

\section{Acknowledgments}
I would like to thank FairfieldNodal for providing the first field data for this study and the opportunity for a summer internship. I also thank Sergey Fomel, Josef Paffenholz, and Araz Mahdad for inspiring discussions and helpful suggestions. Publication was authorized by the director of the Bureau of Economic Geology.

\bibliographystyle{seg}
\bibliography{svmf}

\appendix
\section{Appendix: Review of local similarity}
\cite{fomel2007localattr} defined local similarity between vectors $\mathbf{a}$ and $\mathbf{b}$ as:
\begin{equation}
\label{eq:local}
\mathbf{c}=\sqrt{\mathbf{c}_1^T\mathbf{c}_2}
\end{equation}
where $\mathbf{c}_1$ and $\mathbf{c}_2$ come from two least-squares minimization problem:
\begin{align}
\label{eq:local1}
\mathbf{c}_1 &=\arg\min_{\mathbf{c}_1}\Arrowvert \mathbf{A}-\mathbf{C}_1 \mathbf{B} \Arrowvert_2^2 \\
\label{eq:local2}
\mathbf{c}_2 &=\arg\min_{\mathbf{c}_2}\Arrowvert \mathbf{B}-\mathbf{C}_2 \mathbf{A} \Arrowvert_2^2
\end{align}
where $\mathbf{A}$ is a diagonal operator composed from the elements of $\mathbf{a}$, $\mathbf{B}$ is a diagonal operator composed from the elements of $\mathbf{b}$, and $\mathbf{C}_i$ is a diagonal operator composed from the elements of $\mathbf{c}_i$.
LS problems \ref{eq:local1} and \ref{eq:local2} can be solved with the help of shaping regularization with a local-smoothness constraint:
\begin{align}
\label{eq:local3}
\mathbf{c}_1 &= [\lambda_1^2\mathbf{I} + \mathbf{S}(\mathbf{B}^T\mathbf{B}-\lambda_1^2\mathbf{I})]^{-1}\mathbf{SB}^T\mathbf{a},\\
\label{eq:local4}
\mathbf{c}_2 &= [\lambda_2^2\mathbf{I} + \mathbf{S}(\mathbf{A}^T\mathbf{A}-\lambda_2^2\mathbf{I})]^{-1}\mathbf{SA}^T\mathbf{b},
\end{align}

where $\mathbf{S}$ is a smoothing operator and $\lambda_1$ and $\lambda_2$ are two parameters controlling the physical dimensionality and enabling fast convergence when inversion is implemented iteratively. These two parameters can be chosen as $\lambda_1  = \Arrowvert\mathbf{B}^T\mathbf{B}\Arrowvert_2$ and $\lambda_2  = \Arrowvert\mathbf{A}^T\mathbf{A}\Arrowvert_2$.

