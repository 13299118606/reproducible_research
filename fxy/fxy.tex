

\title{Non-stationary predictive filtering for seismic random noise suppression - A tutorial}
%\author{Hang Wang\footnotemark[1], Wei Chen\footnotemark[2], Weilin Huang\footnotemark[3], Shaohuan Zu\footnotemark[4], Xingye Liu\footnotemark[1], Liuqing Yang\footnotemark[2] and Yangkang Chen\footnotemark[1]}
\author{Authors}
% new functions: 
% 1. 2D + 3D + 4D
% 2. smoothing along frequency 
% 3. frequency-dependent smoothing along space (increase with frequency to better cope with noise)
% 4. User-defined smoothing (radius) along frequency, space (X/Y)
% 5. windowing strategy

% can be separated into two papers (1: with smoothing along frequency and frequency-dependent smoothing, public; 2: all functions, private)

\renewcommand{\thefootnote}{\fnsymbol{footnote}}


\ms{GEO-2020} %\ms{GJI-2019}

%\address{
%\footnotemark[1]
%School of Earth Sciences\\
%Zhejiang University\\
%Hangzhou, Zhejiang Province, China, 310027\\
%yangkang.chen@zju.edu.cn \\
%\footnotemark[2] Key Laboratory of Exploration Technology for Oil and Gas Resources of Ministry of Education\\
%Yangtze University\\
%Daxue Road No.111\\
%Caidian District\\
%Wuhan, China, 430100\\
%\footnotemark[3] State Key Laboratory of Petroleum Resources and Prospecting \\
%China University of Petroleum \\
%Fuxue Road 18th\\
%Beijing, China, 102200\\
%cup\_hwl@126.com \\
%\footnotemark[4] College of Geophysics\\
%Chengdu University of Technology \\
%Dongsanlu, Erxianqiao, Chengdu 610059, Sichuan, China
%%
%%\footnotemark[3] Hubei Cooperative Innovation Center of Unconventional Oil and Gas\\
%%Daxue Road No.111\\
%%Caidian District\\
%}

%\lefthead{Wang et al.}
\righthead{Non-stationary predictive filtering}

\begin{abstract}
Predictive filtering in the frequency domain is one of the most widely used denoising algorithms in the seismic data processing workflow. Predictive filtering is based on the assumption of linear/planar events in the time-space domain. In traditional predictive filtering method, the predictive filter is fixed across the spatial dimension, which cannot deal with the spatial variation of seismic data well. To handle the curving events, the predictive filter is either applied in local windows or extended to a non-stationary version. The regularized non-stationary autoregression (RNAR) method can be treated as a non-stationary extension of the traditional predictive filtering, where the predictive filter coefficients are variable in different space locations. The highly under-determined inverse problem is solved by shaping regularization with a smoothness constraint in space. We further extend the RNAR method to a more general case, where we can apply more constraints to the filter coefficients according to the features of seismic data. First, apart from the smoothness in space, we also apply a smoothing constraint in frequency, considering the coherency of the coefficients in the frequency dimension. Secondly, we apply a frequency dependent smoothing radius along the space dimension to better take advantage of the non-stationarity of seismic data in the frequency axis, and to better deal with noise. The proposed method is validated via several synthetic and field data examples.
\end{abstract}

%\section{Keywords}
%key1,key2,key3

\section{Introduction}
Removing spatially incoherent noise from seismic data is one important task in reflection seismology. The noise suppression is either applied in pre-stack seismic data for improving the velocity analysis and seismic migration to obtain a better seismic image, or applied in post-stack seismic images or attribute maps to facilitate a more reliable seismic interpretation. Therefore, the noise suppression performance highly affects most steps in the seismic processing, imaging, and interpretation chain \cite[]{mostafa2011,mostafa2012,yangkang2018eseis,chenwei2018,chenwei2020}. However, due to the complexity of seismic data, a simple assumption that is required in most denoising algorithms is not always valid. For such reason, many denoising algorithms based on different assumptions are developed during the past decades. 

Those widely used denoising approaches are the prediction-based methods \cite[]{canales1984,siwei2015}, either in $t-x/y$ domain \cite[]{abma1995} or $f-x/y$ domain \cite[]{guochang2012,guochang2013}, which assume that seismic data are spatially predictable and thus a prediction filter can be designed to predict the useful signals. Sparsity based methods first transform the seismic data to a sparsity-promoting domain, where the useful signals and noise can be easily separated because of the amplitude difference. The performance of the transform methods highly depends on the sparsity that is close related with the amplitude difference between signal and noise in the transform domain \cite[]{amir2017,amir2017geo}. Thus, researchers were competing to find the sparest transform for seismic data among curvelet transform \cite[]{neelamani2008}, shearlet transform \cite[]{zhang2018multicomponent}, wavelet transform \cite[]{mostafa2016geo}, Radon transform \cite[]{wenkai2013}, and seislet transform \cite[]{fomel2010seislet,liuyang2010oc,shuwei20163}, etc. The rank-reduction methods emerged in the past decade because of its decent performance in dealing with complex seismic data \cite[]{mssa,weilin2016dmssa,yangkang2016irr5d,wang2020low}. Seismic data are first transformed to a low-rank space and then the principal components are extracted to stand for the useful seismic signals. However, these methods suffer from the intense computation required by the singular value decomposition (SVD) operations. Decomposition based methods divide the seismic data into several components, where the useful signals can be obtained by recombine these components by either frequency or morphological difference \cite[]{weilin2017gji}. Because of the extreme complexity of seismic data, one or more methods are sometimes combined to output a better performance than an individual method \cite[]{yangkang20141}. 

While many denoising algorithms have been developed by researchers , the most widely accepted method in the industry is still the prediction filter based method. Because of the high efficiency and the stable performance (although may not be optimal), processing seismologists prefer to improve the traditional prediction filter by simple and efficient strategies, with the purpose of maintaining the high-efficiency advantage and improving the existing performance, rather than utilizing more complicated and time-consuming denoising methods. The most classic prediction filter originated by \cite{canales1984} with a design of frequency-domain wiener filter, which was referred to as the $f-x$ predictive filtering method. The predictive filter was then introduced separately by \cite{gulunay1986fxdecon} and was referred to as the famous FXDECON algorithm. Since then, a variety of variants have been proposed to improve the classic implementation of the predictive filter. \cite{fxydecon} proposed the non-causal prediction filter to better utilize the constraints from both forward and backward prediction, and applied it to 3D random noise suppression. \cite{naghizadeh2009f} developed an adaptive predictive filter to handle the multi-dip problem in complex seismic data without the need of local windowing and applied it to the seismic interpolation problem.  \cite{guochang2012} developed a regularized non-stationary autoregression (RNAR) method to deal with the spatial curvature and the non-stationarity of seismic data. The RNAR method formulates the predictive filter as the form of auto-regression and extends the autoregression to each non-stationary version. \cite{guochang2013} further applied the RNAR method to 3D problems.  \cite{fx2017} discussed the signal leakage issues in various implementations of the $f-x$ deconvolution algorithms based on the open-source Teapot domain dataset. Recently, \cite{fuchao2020} developed a structure-guided predictive filter to handle the non-stationarity of seismic data by squeezing and shrink the length of prediction filter according to the structural complexity. 
 

In this paper, we further extend the RNAR approach to a more integrated framework, where we deal with the spatial non-stationarity of seismic data by extending the stationary autoregression to non-stationary autoregression and take advantage of the temporal coherency, i.e., narrow-band and smooth spectrum, by smoothing the autoregression coefficients in frequency and applying frequency-dependent smoothing to the coefficients in space.  To distinguish from the RNAR method, we name the proposed new method as the non-stationary predictive filtering (NPF) method. Compared with the stationary predictive filtering (PF) \cite[]{canales1984}, there are two meanings of the non-stationarity we refer to here. First, we synthesize the non-stationary data by regularized non-stationary autoregression. Secondly, we deal with the non-stationary model (the frequency-dependent non-stationary autoregression coefficients) by applying non-stationary model constraint, e.g., frequency-dependent spatial smoothing. We organize the paper as follows. We first introduce the fundamentals of the stationary predictive filtering (SPF) and NPF, with the focus on the new constraints we have applied compared with the RNAR method. Then we use several 2D/3D synthetic and real data examples to demonstrate the better denoising performance of the proposed NPF method. Finally, we sumarize some conclusions according to the numerical experiments.


\section{Theory}
\subsection{Stationary predictive filtering}
The stationary predictive filtering (SPF) is based on the plane-wave assumption. Let us first consider the plane-wave equation as follows \cite[]{fomel2002pwd}:
\begin{equation}
\label{eq:plane}
p \frac{\partial u}{\partial t} + \frac{\partial u}{\partial x} = 0,
\end{equation}
where $p$ denotes the slope, $u(t,x)$ denotes the wavefield at time $t$ and space $x$. The plane-wave equation has the following analytical solution:
\begin{equation}
\label{eq:pre}
u(t,x)=g(t-px),
\end{equation}
where $g(t)$ denotes the wavelet.  When taken into the frequency domain:
\begin{equation}
\label{eq:pref}
U(f,x)=G(f)e^{i2\pi fpx},
\end{equation}
where $U(f,x)$ and $G(f)$ denote the frequency-domain spectrum of wavefield and wavelet. From equation \ref{eq:pref}, it can be derived that:
\begin{equation}
\label{eq:pref2}
U(f,x+1)=e^{i2\pi fp}U(f,x).
\end{equation}
When several planes exist, 
\begin{equation}
\label{eq:pref22}
U(f,x+1)=e^{i2\pi fp_1}U(f,x)+e^{i2\pi fp_2}U(f,x-1)+e^{i2\pi fp_3}U(f,x-2)+\cdots+e^{i2\pi fp_{L}}U(f,x-L+1),
\end{equation}
which can be considered as an auto-regression (AR) model or order $L$ \cite[]{tufts1980,canales1984,yangkang20141}. The AR model expressed in equation \ref{eq:pref22} is known as the $f-x$ predictive filtering based on the forward prediction. The AR model can also be formulated as a causal prediction filter:
\begin{equation}
\label{eq:pref22causal}
U(f,x)=\sum_{i=-L,i\ne 0}^{L}a_i(f)U(f,x+i),
\end{equation}
where $a_i(f)=e^{i2\pi fp_i}$. Here, $L$ denotes the order of the causal prediction filter. 

\subsection{Non-stationary predictive filtering}
Since the SPF is based on the plane-wave assumption, it is normally combined with a local windowing strategy to best denoise a complicated seismic data. However, there will be many drawbacks using the local windowing strategy, e.g., one needs to consider the window size, the tapering strategy around edges, the parameter inconsistency issue \cite[]{shaohuan2017gji}. The more appropriate way to deal with complicated seismic data is to modify the SPF expression in equation \ref{eq:pref22} in a non-stationary form:
\begin{equation}
\label{eq:ncausal}
U(f,x)=\sum_{i=-L,i\ne 0}^{L}a_i(f,x)U(f,x+i),
\end{equation}
where the auto-regression coefficients $a_i(f,x)$ vary across the $f-x$ domain. The auto-regression formula expressed in equation \ref{eq:ncausal} bypass the need of local windowing since it expresses the auto-regression coefficients, which is related with the slope, in an non-stationary version. However, equation \ref{eq:ncausal} requires solving a highly under-determined inverse problem. Traditionally, the stationary \cite[]{canales1984} or non-stationary auto-regression coefficients \cite[]{guochang2012}, $a_i(f)$ or $a_i(f,x)$, are solved frequency-by-frequency. For example, for the non-stationary auto-regression in equation \ref{eq:ncausal}, $a_i(f,x)$ is obtained by solving the following equation system:
\begin{equation}
\label{eq:sys}
\begin{split}
\hat{a_i}(f_1,x) &= \arg \min_{a_i(f_1,x)} \parallel U(f_1,x)-\sum_{i=-L,i\ne 0}^{L}a_i(f_1,x)U(f_1,x+i) \parallel  \\
\hat{a_i}(f_2,x) &= \arg \min_{a_i(f_2,x)} \parallel U(f_2,x)-\sum_{i=-L,i\ne 0}^{L}a_i(f_2,x)U(f_2,x+i) \parallel  \\
&\vdots\\
\hat{a_i}(f_{N_f},x) &= \arg \min_{a_i(f_{N_f},x)} \parallel U(f_{N_f},x)-\sum_{i=-L,i\ne 0}^{L}a_i(f_{N_f},x)U(f_{N_f},x+i) \parallel.
\end{split}
\end{equation}
Solving equation \ref{eq:sys} is referred to as the regularized non-stationary autoregression \cite[]{guochang2012}. The benefit of solving the non-stationary auto-regression coefficients $a_i(f,x)$ frequency by frequency via equation \ref{eq:sys} is that the computation is easy to parallelized, as introduced in \cite{guochang2012}. However, solving $a_i(f,x)$ frequency by frequency could lose the inherent model constraint along the frequency axis. 

Here, we propose to formulate the inverse problem \ref{eq:ncausal} in a more integrated form, where we can understand the nature of this inverse problem more intuitively, thereby applying more physically plausible constraints to the model, i.e., $a_i(f,x)$. To solve equation \ref{eq:ncausal} using the classic inversion theory, we first transform it into a matrix-vector form as follows:
\begin{align}
\label{eq:mv}
\mathbf{u} = \left[\begin{array}{cccc}
\mathbf{u}_{-L} &\mathbf{u}_{-L+1}&\cdots&\mathbf{u}_{L}
\end{array}\right]\left[\begin{array}{c}
\mathbf{a}_{-L}\\
\mathbf{a}_{-L+1}\\
\vdots\\
\mathbf{a}_L
\end{array}\right],
\end{align}
where $\mathbf{u}$ is the vector form of $U(f,x)$, $\mathbf{u}_i$ is the vector form of $U(f,x+i)$, and $\mathbf{a}_i$ is the vector form of $a_i(f,x)$. All $\mathbf{u}$, $\mathbf{u}_i$, and $\mathbf{a}_i$ are vectors of size $N_f\times N_x$. $N_f$ and $N_x$ denote the size in frequency and space dimensions. Equation \ref{eq:mv} can be expressed more briefly as
\begin{equation}
\label{eq:fx}
\mathbf{Fa}=\mathbf{u},
\end{equation}
where
\begin{equation}
\label{eq:fx2}
\mathbf{F}=\left[\begin{array}{cccc}
\mathbf{u}_{-L} &\mathbf{u}_{-L+1}&\cdots&\mathbf{u}_{L}
\end{array}\right] \quad \text{and}\quad \mathbf{a}=\left[\begin{array}{c}
\mathbf{a}_{-L}\\
\mathbf{a}_{-L+1}\\
\vdots\\
\mathbf{a}_L
\end{array}\right].
\end{equation}
In the form of an optimization problem, equation \ref{eq:fx} can be solved via:
\begin{equation}
\label{eq:fx3}
\hat{\mathbf{a}} = \arg\min_{\mathbf{a}} \parallel \mathbf{u}-\mathbf{Fa} \parallel_2^2 + \mathbf{R}(\mathbf{a}),
\end{equation}
where $\mathbf{R}$ denotes a regularization operator.  Equation \ref{eq:fx3} can be solved using standard regularized inversion methods, e.g., Tikhonov regularization \cite[]{tikhonov1963} or shaping regularization \cite[]{fomel2007shape}. Here, we use the shaping regularization method to solve the optimization problem \ref{eq:fx3} via an iterative procedure:
 \begin{equation}
\label{eq:iter}
\mathbf{a}_{m+1} = \mathbf{S}[\mathbf{a}_m + \mathbf{B}^T(\mathbf{u}-\mathbf{Fa}_m)] ,
\end{equation}
where $\mathbf{a}_{m}$ denotes the model after $m$th iteration. $\mathbf{S}$ and $\mathbf{B}$ are called shaping and backward operators, $\mathbf{S}$ is used to apply a priori constraint to the model while  $\mathbf{B}$ provides an approximate inverse of the forward operator $\mathbf{F}$. In the conjugate gradient algorithm (CG) \cite[]{fomel2007shape}, the model update $\mathbf{B}^T(\mathbf{u}-\mathbf{Fa}_m)$ is calculated following the general CG definition. The shaping operator $\mathbf{S}$ is chosen as a multi-dimensional triangle smoothing operator, i.e., smoothing along the space direction (like in \cite{guochang2012}) as well as the frequency direction to ensure the frequency smoothness. The iterative formula \ref{eq:iter} converges to the following solution:
\begin{equation}
\label{eq:fx33}
\hat{\mathbf{a}} = [\lambda^2\mathbf{I} + \mathbf{S}(\mathbf{F}^T\mathbf{F}-\lambda^2\mathbf{I}) ]^{-1}\mathbf{S}\mathbf{F}^T\mathbf{a},
\end{equation}
where $\lambda$ is fixed to be $\parallel\mathbf{F}^T\mathbf{F}\parallel_2$ \cite[]{fomel2007shape}. Because of the convenience offerred by the shaping regularization, it is flexible to apply any physical meaningful constraint to the model, e.g., a frequency-dependent smoothing along the space direction. From equation \ref{eq:fx}, we are aware that the model in the inverse problem corresponds to the non-stationary auto-regression coefficients spreading across the frequency-space domain. It is intuitive that the frequency-space coefficients should be smooth in both frequency and space dimensions, thus we apply the smoothing in both directions. It is also intuitive that the auto-regression coefficients could be more and more oscillating and even unstable when frequency increases, since high-frequency spectrum more corresponds to the noise. To facilitate a stronger anti-noise ability and to ensure the stability when solving high-frequency non-stationary auto-regression coefficients, we propose to apply the frequency-dependent smoothing. For example, we can increase the smoothing radius from a low value around the dominant frequency (e.g., 10 Hz) to a larger value around the Nyquist frequency following a fraction power function:
\begin{align}
\label{eq:fs}
R(f) = \left\{\begin{array}{ll}
R_d, & 0<=f<=f_d \\
R_d+\left(\left(\frac{f-f_d}{f_n-f_d}\right)^{\alpha}R_d-R_n\right),& f_d<f<=f_n
\end{array} \right.
\end{align}
where $R_d$ and $R_n$ are the smoothing radii for dominant ($f_d$) and Nyquist ($f_n$) frequencies, and $\alpha$ is called the fraction parameter ($\alpha=0.5$ in default). To distinguish between the regularized non-stationary auto-regression (RNAR) method and the proposed method (although both are intended to solve the non-stationary auto-regression coefficients), we name the proposed method the non-stationary predictive filtering (NPF) method. We admit that RNAR and NPF are closely related with each other, but with differences in the implementation and model constraint. These differences, however, could result in significant differences in final denoised result, which will be illustrated in the section of EXAMPLES. 

\subsection{3D extension of the non-stationary predictive filtering}
The non-stationary AR model can be extended to 3D or even higher dimensions straightforwardly. Considering a 3D seismic data $u(t,x,y)$, first we transform the 3D data volume from $t-x-y$ domain to $f-x-y$ domain, $U(f,x,y)$, then we can express the non-stationary AR model as:
\begin{equation}
\label{eq:ncausal3}
U(f,x,y)=\sum_{iy=-L_y,iy\ne 0}^{L_y}\sum_{ix=-L_x,ix\ne 0}^{L_x}a_{ix,iy}(f,x,y)U(f,x+ix,y+iy),
\end{equation}
where $L_x$ and $L_y$ denote the AR orders in $x$ and $y$ directions, respectively. 
Equation \ref{eq:ncausal3} can be expressed briefly in the form of equation \ref{eq:fx} but with $\mathbf{F}$ and $\mathbf{a}$ denoting slightly different formulas:
\begin{equation}
\label{eq:fx3d}
\mathbf{F}=\left[\begin{array}{cccc}
\tilde{\mathbf{u}}_{-Ly} &\tilde{\mathbf{u}}_{-Ly+1}&\cdots&\tilde{\mathbf{u}}_{Ly}
\end{array}\right] \quad \text{and}\quad \mathbf{a}=\left[\begin{array}{c}
\tilde{\mathbf{a}}_{-L_y}\\
\tilde{\mathbf{a}}_{-L_y+1}\\
\vdots\\
\tilde{\mathbf{a}}_{L_y}
\end{array}\right],
\end{equation}
where
\begin{equation}
\label{eq:fx3d}
\tilde{\mathbf{u}}=\left[\begin{array}{cccc}
\mathbf{u}_{-{L_x}} &\mathbf{u}_{-L_x+1}&\cdots&\mathbf{u}_{L_x}
\end{array}\right] \quad \text{and}\quad \tilde{\mathbf{a}}=\left[\begin{array}{c}
\mathbf{a}_{-L_x}\\
\mathbf{a}_{-L_x+1}\\
\vdots\\
\mathbf{a}_{L_x}
\end{array}\right].
\end{equation}
In 3D case, the size of vector $\mathbf{a}$ is $(L_xL_y-1)N_fN_xN_y$, where $N_x$ and $N_y$ denote the size in the x and y dimensions. Solving the non-stationary AR problem in 3D is the same as the 2D case based on the shaping regularization method expressed in equation \ref{eq:iter}. However, in the 3D case, the multi-dimensional triangle smoothing operator means smoothing in frequency, space X and space Y directions.  The smoothing radius formula expressed in equation \ref{eq:fs} holds for the smoothing radii in both X and Y directions. 

\section{Examples}
We use one synthetic and two field data examples to demonstrate the performance of the NPF method, and compare it with the SPF and RNAR methods. For the synthetic example, we use the signal-to-noise ratio (SNR) as the metric to evaluate the denoising performance. The definition of SNR is given as:
\begin{equation}
\label{eq:diff}
SNR=10\log_{10}\frac{\Arrowvert \mathbf{s} \Arrowvert_2^2}{\Arrowvert \hat{\mathbf{s}} -\mathbf{s}\Arrowvert_2^2},
\end{equation}
where $\mathbf{s}$ is the clean signal and $\hat{\mathbf{s}}$ is the noisy/denoised data. In addition, we use local similarity \cite[]{fomel2007localattr,yangkang2015ortho} to evaluate the signal leakage of different denoising methods. For the real data examples, we only use the local similarity to evaluate the signal leakage. The local similarity between the denoised data and removed noise is calculated to indicate which part of the noise section is significantly similar to the same part of the denoised data, i.e., signal leakage.

To ensure all the competing methods to obtain acceptable results, we use the local similarity as a reference to tune the parameters so that each method can obtain the same level of signal preservation (i.e., low signal leakage). Based on the same level of signal leakage, we focus on the comparison of denoised data. The cleaner and smoother seismic data indicate better performance. For those methods that will inevitably cause high signal leakage, we tune the parameters to minimize the leakages as revealed from the local similarity metric.
 
The synthetic example is plotted in Figure \ref{fig:cmp0,cmp}, with left figure showing the clean data and right showing the noisy data. The SNR of the noisy data is -9.23 dB. Figure \ref{fig:test2,test1,test0,test00} shows the denoised data using different methods. Figure \ref{fig:test2} shows the denoised data using the SPF method. The SPF method uses a causal autoregression filter with two points prediction length for forward and backward prediction. The prediction length for X and Y directions are the same. Figure \ref{fig:test1} shows the result from the RNAR method. Figure \ref{fig:test0} shows the result from the NPF method (with only frequency-direction smoothing but without frequency-dependent smoothing). Figure \ref{fig:test00} plots the result from the NPF method with both frequency-direction smoothing and frequency-dependent space-direction smoothing (NPFFS). For both RNAR and NPF, we use a causal filter with two-points forward and backward prediction in each space direction (X and Y). We use a smoothing radius of three points in frequency for NPF (Figures \ref{fig:test0} and \ref{fig:test00}).  For frequency-dependent smoothing, we increase the smoothing radius from four points around the dominant frequency (e.g., 10 Hz for this example) to 20 points of the Nyquist frequency following a fraction power function (with the power fraction to be 0.5). This frequency smoothing strategy is the default form of the NPF method. Figure \ref{fig:test2dif,test1dif,test0dif,test00dif} shows the removed noise corresponding to the four approaches shown in Figure \ref{fig:test2,test1,test0,test00}.  It is clear that only the SPF method causes obvious signal damages, while all other methods do not leave significant signal energy in the noise. The SNRs of the four methods, namely SPF, RNAR, NPF, NPFFS, are 
2.34 dB, 2.56 dB, 2.93 dB, and 4.80 dB, respectively, indicating a gradually improved performance from the SPF to RNAR and to NPF methods. The frequency-space domain spectra comparison is plotted in Figure \ref{fig:hyper-f,hypern-f,test2-f,test1-f,test0-f,test00-f}. It is clear that the noise level is very high from the noise spectra in Figure \ref{fig:hypern-f}. The SPF method significantly damages the signal spectra, as shown in Figure \ref{fig:test2-f}. The RNAR method obtains a good recovery of the signal spectra. Both NPF methods obtain cleaner spectra than the RNAR method. The NPF with frequency smoothing obtains the cleanest signal spectra.
It is convenient to compare the signal leakage by the local similarity maps, as plotted in Figure \ref{fig:simi2,simi1,simi0,simi00}. The signal leakage of the SPF method is clearly revealed in Figure \ref{fig:simi2} as the high similarity values depict the distribution of the seismic events correctly. Both RNAR and NPFFS methods cause negligible signal leakage, while the NPF (without frequency-dependent smoothing) causes a little stronger signal leakages. However, the large similarity values are mostly spreading those areas lacking signals, indicating that the pure frequency-direction smoothing could cause potential hidden weak artifacts (which is not visible from the data) and further cause the weak artifacts in the noise section. Such hidden artifacts could result in an abnormal local similarity value. However, note that since the overall similarity values of all the methods are very low, e.g., below 0.1, the aforementioned signal leakages for these methods are not significant. A single-trace comparison (20th X-direction trace and 20th Y-direction trace) of this dataset is plotted in Figure \ref{fig:hyper-ss-0,hyper-ss-z}. The top panel shows the comparison in the original scale and the bottom row plots the comparison in the zoomed scale. The zooming area is indicated by the frame box in the top panel. In Figure \ref{fig:hyper-ss-0,hyper-ss-z}, the black line corresponds to the clean data. The green line corresponds to the noisy data. The black line corresponds to the clean data. The pink line corresponds to the SPF method. The red line corresponds to the RNAR method.  The cyan and blue lines correspond to the denoised data using the NPF methods without and with  frequency-dependent smoothing, respectively. It is clear that the SPF deviates from the clean trace most. The RNAR method contains the largest residual noise. Both NPF methods are close to the clean trace but the NPF method with frequency-dependent smoothing is closest to the clean trace.


The first field data example is a 2D seismic section, which has been previously used in \cite{guochang2012}. It is a very complicated seismic profile after time-migration, but is still very noisy, as shown in Figure \ref{fig:real2d-0}. Figure \ref{fig:r2test2-0,r2test1-0,r2test0-0,r2test00-0} plots the comparison of the four methods, i.e., SPF, RNAR, NPF, NPFFS. Comparing all four results with the raw data, it is clear that all these methods obtain significant improvement. The removed noise sections are plotted in Figure \ref{fig:r2test2-n-0,r2test1-n-0,r2test0-n-0,r2test00-n-0}, where the SPF method causes some visible damages in the removed noise and other three methods cause negligible damages. The local similarity comparison is plotted in Figure \ref{fig:r2simi2,r2simi1,r2simi0,r2simi00}, where all the methods cause low local similarity maps. Among them, the SPF, RNAR, and NPF methods have slightly higher local similarity. NPFFS method shows obviously lower similarity values, indicates a better signal preservation because of the frequency-dependent smoothing. For a better comparison of the denoised results, we zoom in an area from each plot in Figures \ref{fig:real2d-0} and \ref{fig:r2test2-0,r2test1-0,r2test0-0,r2test00-0}, and show them in Figure \ref{fig:real2d-z,r2test2-z,r2test1-z,r2test0-z,r2test00-z,r2test00dif-z}. From the zoomed sections, we confirm that the SPF method loses some details, the RNAR method contains more residual noise, the NPF method obtains quite successful denoising performance, while the NPFFS method obtains an even better result than NPF by mitigating more high-frequency noise. The difference between Figures \ref{fig:r2test0-z} and \ref{fig:r2test00-z} is plotted in Figure \ref{fig:r2test00dif-z}, where it shows some removed high-frequency noise due to the frequency-dependent smoothing. We further zoom in a part from the removed noise sections (Figure \ref{fig:r2test2-n-0,r2test1-n-0,r2test0-n-0,r2test00-n-0}) and show them in Figure \ref{fig:r2test2-n-z,r2test1-n-z0,r2test0-n-z,r2test00-n-z}, where we can see more clearly that both SPF and RNAR methods cause some leaked signals and the NPF method does not cause visible signal leakage. 

The third example is a 3D field dataset, as plotted in Figure \ref{fig:real}. The denoised data using different methods are plotted in Figure \ref{fig:rtest2,rtest1,rtest0,rtest00}. The SPF and NPF methods seem to obtain smoother results than the RNAR method. The removed noise cubes are plotted in Figure \ref{fig:rtest2-n,rtest1-n,rtest0-n,rtest00-n}. The local similarity cubes are plotted in Figure \ref{fig:rsimi2,rsimi1,rsimi0,rsimi00}, showing the a similar level of signal leakage for all the methods. It is also clear that the RNAR method seems to cause a little more signal leakages than the other methods. Figure \ref{fig:real-z,rtest2-z,rtest1-z,rtest0-z,rtest00-z,rtest00dif-z} shows a zoomed comparison between the raw data and denoised data using different methods. From the zoomed comparison, we conclude that the result from the SPF method is smooth, but contains a little weaker energy. The RNAR method removes less noise. The two NPF approaches obtain smooth results and preserves the details well. The frequency-dependent smoothing can make the denoised result a little smoother by removing more high-frequency noise. 

\inputdir{hyper3d}
\multiplot{2}{cmp0,cmp}{width=0.45\textwidth}{Synthetic example. (a) Clean data. (b) Noisy data (SNR=-9.23 dB).}

\multiplot{4}{test2,test1,test0,test00}{width=0.45\textwidth}{Denoised data using (a) stationary predictive filtering (SNR=2.34 dB), (b) regularized non-stationary auto-regression (SNR=2.56 dB), (c) non-stationary predictive filtering without frequency-dependent smoothing (SNR=2.93 dB), and (d) non-stationary predictive filtering with frequency-dependent smoothing (SNR=4.80 dB).}
\multiplot{4}{test2dif,test1dif,test0dif,test00dif}{width=0.45\textwidth}{Removed noise using (a) stationary predictive filtering, (b) regularized non-stationary auto-regression, (c) non-stationary predictive filtering without frequency-dependent smoothing, and (d) non-stationary predictive filtering with frequency-dependent smoothing.}

\multiplot{6}{hyper-f,hypern-f,test2-f,test1-f,test0-f,test00-f}{width=0.3\textwidth}{FX spectra of (a) clean data, (b) noisy data, (c) denoised data using stationary predictive filtering, (d) denoised data using regularized non-stationary auto-regression, (e) non-stationary predictive filtering without frequency-dependent smoothing, and (f) non-stationary predictive filtering with frequency-dependent smoothing.}

\multiplot{4}{simi2,simi1,simi0,simi00}{width=0.45\textwidth}{Local similarity (between denoised data and removed noise) using (a) stationary predictive filtering, (b) regularized non-stationary auto-regression, (c) non-stationary predictive filtering without frequency-dependent smoothing, and (d) non-stationary predictive filtering with frequency-dependent smoothing.}

\multiplot{2}{hyper-ss-0,hyper-ss-z}{width=0.8\textwidth}{Single-trace comparison (20th X-direction trace and 20th Y-direction trace). (a) Comparison in the original scale. (b) Comparison in the zoomed scale. The zooming area is highlighted by the black frame box in (a). The green line corresponds to the noisy data. The black line corresponds to the clean data. The pink line corresponds to the SPF method. The red line corresponds to the RNAR method.  The cyan and blue lines correspond to the denoised data using the NPF methods without and with  frequency-dependent smoothing, respectively. }

\inputdir{real2d_win}
\plot{real2d-0}{width=0.8\textwidth}{Real 2D seismic profile.}

\multiplot{4}{r2test2-0,r2test1-0,r2test0-0,r2test00-0}{width=0.4\textwidth}{Denoised data using (a) stationary predictive filtering, (b) regularized non-stationary auto-regression, (c) non-stationary predictive filtering without frequency-dependent smoothing, and (d) non-stationary predictive filtering with frequency-dependent smoothing.}
\multiplot{4}{r2test2-n-0,r2test1-n-0,r2test0-n-0,r2test00-n-0}{width=0.4\textwidth}{Removed noise using (a) stationary predictive filtering, (b) regularized non-stationary auto-regression, (c) non-stationary predictive filtering without frequency-dependent smoothing, and (d) non-stationary predictive filtering with frequency-dependent smoothing.}

\multiplot{4}{r2simi2,r2simi1,r2simi0,r2simi00}{width=0.4\textwidth}{Local similarity (between denoised data and removed noise) using (a) stationary predictive filtering, (b) regularized non-stationary auto-regression, (c) non-stationary predictive filtering without frequency-dependent smoothing, and (d) non-stationary predictive filtering with frequency-dependent smoothing.}

\multiplot{6}{real2d-z,r2test2-z,r2test1-z,r2test0-z,r2test00-z,r2test00dif-z}{width=0.4\textwidth}{Zoomed comparison of denoised data using (a) raw data, (b) stationary predictive filtering, (c) regularized non-stationary auto-regression, (d) non-stationary predictive filtering without frequency-dependent smoothing, and (e) non-stationary predictive filtering with frequency-dependent smoothing. (f) Difference between (d) and (e). Note that the frequency-dependent smoothing (e) has removed more high-frequency noise compared with (d).}

\multiplot{4}{r2test2-n-z,r2test1-n-z0,r2test0-n-z,r2test00-n-z}{width=0.4\textwidth}{Zoomed comparison of removed noise using (a) stationary predictive filtering, (b) regularized non-stationary auto-regression, (c) non-stationary predictive filtering without frequency-dependent smoothing, and (d) non-stationary predictive filtering with frequency-dependent smoothing. Note the obvious signal leakages in (a) and (b). }


\inputdir{real}
\plot{real}{width=0.8\textwidth}{3D field data example. }

\multiplot{4}{rtest2,rtest1,rtest0,rtest00}{width=0.4\textwidth}{Denoised data using (a) stationary predictive filtering, (b) regularized non-stationary auto-regression, (c) non-stationary predictive filtering without frequency-dependent smoothing, and (d) non-stationary predictive filtering with frequency-dependent smoothing.}
\multiplot{4}{rtest2-n,rtest1-n,rtest0-n,rtest00-n}{width=0.4\textwidth}{Removed noise using (a) stationary predictive filtering, (b) regularized non-stationary auto-regression, (c) non-stationary predictive filtering without frequency-dependent smoothing, and (d) non-stationary predictive filtering with frequency-dependent smoothing.}

\multiplot{4}{rsimi2,rsimi1,rsimi0,rsimi00}{width=0.4\textwidth}{Local similarity (between denoised data and removed noise) using (a) stationary predictive filtering, (b) regularized non-stationary auto-regression, (c) non-stationary predictive filtering without frequency-dependent smoothing, and (d) non-stationary predictive filtering with frequency-dependent smoothing.}

\multiplot{6}{real-z,rtest2-z,rtest1-z,rtest0-z,rtest00-z,rtest00dif-z}{width=0.4\textwidth}{Zoomed comparison of denoised data using (a) raw data, (b) stationary predictive filtering, (c) regularized non-stationary auto-regression, (d) non-stationary predictive filtering without frequency-dependent smoothing, and (e) non-stationary predictive filtering with frequency-dependent smoothing. (f) Difference between (d) and (e). Note that the frequency-dependent smoothing (e) has removed more high-frequency noise compared with (d).}

%\section{Discussions}
\section{Conclusions}
The traditional predictive filter is not able to characterize the non-stationarity of seismic data along the space dimension. 
The RNAR method calculates the space-variable predictive filter coefficients by solving an inverse problem with spatial smoothness constrained shaping regularization. By formulating the non-stationary filter coefficients in all frequency bands into the same inverse problem, we can solve 
a non-stationary predictive filter that is constrained in both space and frequency dimensions. The new non-stationary predictive filtering method is more competent in denoising very noisy seismic data with more robust performance. In addition, the frequency-dependent smoothing method helps the non-stationary predictive filtering method to attenuate more random noise in moderate to high-frequency bands. The proposed method is easily applied to 2D/3D/4D applications with effective performance. Several synthetic and field data examples demonstrate the appealing performance.


\section{DATA AND MATERIALS AVAILABILITY}
All datasets and source codes associated with this research will be made available online (www.ahay.org) in the format of reproducible research \cite[]{mada2013}.

%\section{Acknowledgements}

%\inputdir{test}
%\plot{test1}{width=\textwidth}{Separated x-component of the S1 elastic wavefield in the orthorhombic media.}
%\multiplot{2}{test1,test2}{width=0.45\textwidth}{(a) Caption a. (b) Caption b.}




\bibliographystyle{seg}
\bibliography{fxy}

%\AtEndDocument{}

%\newpage
%\listoffigures



%\begin{figure}[htb!]
%	\centering
%	\subfloat[]{\includegraphics[width=0.8\textwidth]{Fig/fig}}
%	\caption{Caption.}
%	\label{fig:fig}
%\end{figure}

%\begin{figure}[htb!]
%	\centering
%	\subfloat[]{\includegraphics[width=0.45\textwidth]{Fig/fig1}
%    \label{fig:fig1}}\\
%    \subfloat[]{\includegraphics[width=0.45\textwidth]{Fig/fig2}
%    \label{fig:fig2}}\\
%	\caption{(a) Caption a. (b) Caption b.}
%	\label{fig:fig1,fig2}
%\end{figure}

%\begin{figure*}[ht!]
%	\centering
%	\subfloat[]{\includegraphics[width=0.45\textwidth]{Fig/fig1}
%    \label{fig:fig1}}\\
%    \subfloat[]{\includegraphics[width=0.45\textwidth]{Fig/fig2}
%    \label{fig:fig2}}\\
%	\caption{(a) Caption a. (b) Caption b.}
%	\label{fig:fig1,fig2}
%\end{figure*}

%\begin{table}[h]
%\caption{Table caption}
%\begin{center}
%     \begin{tabular}{|c|c|c|c|c|c|} 
%	  \hline Column1 (unit)  & Column2 (unit) & Column3 (unit) \\ 
%	  \hline 1 & 2  & 3 \\
%       \hline
%    \end{tabular} 
%\end{center}
%\label{tbl:table1}
%\end{table}

